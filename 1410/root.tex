\documentclass[submission,copyright,creativecommons]{eptcs}
\providecommand{\event}{UITP 2014}
\usepackage{breakurl}

\usepackage{graphicx}
\usepackage[export]{adjustbox}
\usepackage[utf8]{inputenc}
\usepackage{ifthen}
\usepackage{isabelle,isabellesym}

\hyphenation{Isabelle}
\hyphenation{Scala}

\isadroptag{theory}
\isabellestyle{literal}
%\renewcommand{\isadigit}[1]{\isatext{#1}}

\newcommand{\secref}[1]{\S\ref{#1}}
\newcommand{\figref}[1]{figure~\ref{#1}}
\newcommand{\Figref}[1]{Figure~\ref{#1}}

\renewcommand{\hyperlink}[2]{#2}  %evade macros stemming from entity antiquotations

\sloppy
\hyphenation{Isabelle}

\begin{document}

\title{System description: Isabelle/jEdit in 2014}
\author{Makarius Wenzel \thanks{Research supported by Project
    Paral-ITP (ANR-11-INSE-001).}
\institute{Univ. Paris-Sud, Laboratoire LRI, UMR8623, Orsay, F-91405, France \\
  CNRS, Orsay, F-91405, France}}
\def\titlerunning{Isabelle/jEdit in 2014}
\def\authorrunning{M. Wenzel}
\maketitle

\begin{abstract}
  This is an updated system description for Isabelle/jEdit, according to the
  official release Isabelle2014 (August 2014). The following new PIDE
  concepts are explained: asynchronous print functions and document
  overlays, syntactic and semantic completion, editor navigation,
  management of auxiliary files within the document-model.
\end{abstract}

\input{Paper}

\bibliographystyle{eptcs}
\bibliography{root}

\end{document}

